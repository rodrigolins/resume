% LaTeX source of my resume
% =========================

% Commented for easy reuse... ;)

% See the `README.md` file for more info.

% This file is licensed under the CC-NC-ND Creative Commons license.


% start a document with the here given default font size and paper size
\documentclass[10pt,a4paper]{article}

% include the `tex` instructions that takes care of loading packages and defining commands
\include{resume-commands}



\begin{document}  % begin the content of the document
\sloppy  % this to relax whitespacing in favour of straight margins

\maintitle{Arjun Naik}{October 29, 1987}  % title on top of the document

\nobreakvspace{0.3em}  % add some page break averse vertical spacing

% \noindent prevents paragraph's first lines from indenting
% \mbox is used to obfuscate the email address
% \sbull is a spaced bullet
% \href well..
% \\ breaks the line into a new paragraph
\noindent\href{mailto:Arjun.RN@gmail.com}{Arjun.RN\mbox{}@\mbox{}gmail.com}\sbull
\textsmaller{+}49.17657886406
\sbull arjun-naik \emph{(Skype)}
\\
\sbull
\href{https://de.linkedin.com/in/arnaik}{https://de.linkedin.com/in/arnaik}
\sbull
\href{https://github.com/arjunrn}{https://github.com/arjunrn}
\\
Gret-Palucca-Stra{\ss}e 9, Whng. 1214\sbull
01069\sbull
Dresden\sbull
Germany

\spacedhrule{0.9em}{-0.4em}  % a horizontal line with some vertical spacing before and after

\roottitle{Summary}  % a root section title

\vspace{-1.3em}  % some vertical spacing
\begin{multicols}{2}  % open a multicolumn environment
\noindent \emph{A organised, meticulous, and resourceful software engineering with a focus on building distributed systems.}
\\
It has been 10 years since I picked up a book on C programming. Since then I have been led deeper into the world of software development. My discovery of Python was the second biggest turning point. This sparked my interest in Web applications and subsequently Distributed Systems. 

After obtaining a Bachelor's degree in Computer Science and Engineering I spent 2 years working at a consultancy. Here I got my first experience of participating in large projects and the responsibilities that come along with it. After that I moved to a startup where I built web applications and developed an interest in Android development. After working for 2 years in the industry I decided to continue my education in Germany. Studying at TU Dresden in the Systems Engineering chair has given me a unique opportunity to participate is several diverse research projects.

Once I graduate I hope to work in an environment which encourages independent decision making and offers interesting challenges. Since I am a big proponent of open-source, I also wish that my work both utilises and contributes back to the community.

\end{multicols}

\spacedhrule{0em}{-0.4em}

\roottitle{Experience}

\headedsection  % sets the header for the section and includes any subsections
{\href{http://www.netbramha.com}{Netbramha Studios}}
{\textsc{Bangalore, India}}
{
  \headedsubsection  % sets the header for a subsection and contains usually body text
    {\acr{Web Developer}}
    {Oct 2010 - May 2011}
    {\bodytext{Netbramha Studios is a consultancy which develops well designed and content intensive websites for their clients. Hired right out of college to handle the technical aspects of maintaining these websites as well as minor front-end development. Developed new features using both third-party and in-house frameworks. For the front-end development used jQuery and Dojo and also did some basic layout and styling with \acr{HTML+CSS}. As my responsibilities grew was handed the task of developing a complete web-based front end for one of India's largest pizza delivery chains.}}
}

\headedsection
{\href{http://www.notiphi.com}{Locus Labs}}
{\textsc{Bangalore, India}}
{
 \headedsubsection
 {\acr{Web and Mobile Developer}}
 {Jun 2011 - August 2012}
 {\bodytext{
 Locus Labs was at that time a unseeded startup which developed apps for the Singapore market. The main prodcut {\href{http://www.notikum.com}{Notikum}} found offers and bargains for shoppers based on their profile. This involved collecting data from hundreds of websites. I was in the team which scraped the data and indexed it with Solr. I did this while concurrently developing and maintaining the website for this application. The surge in mobile apps necessitated that an Android app be development which I initially did myself and later coordinated with new hires.

After the startup pivoted was given further responsibilities. Was put in charge of the development of the new product {\href{http://www.notiphi.com}{Notiphi}}. Location based advertising was the primary goal of this service. Designed and implemented the initial prototype. Also developed the tools needed for the deployment and monitoring of the cloud-based infrastructure. Using my previous Android experience developed the Android SDK which could be used by third-party developers to integrate this service into their apps.}}
}

\headedsection
{\href{http://dmfs.org/}{DMFS}}
{\textsc{Dresden, Germany}}
{
 \headedsubsection
 {\acr{Application Developer}}
 {Jan 2013 - March 2014}
 {\bodytext{
    DMFS develops data sync applications for the Android platform. I was responsible for the development of the {\href{https://play.google.com/store/apps/details?id=org.dmfs.tasks}{Tasks}} app which uses the CalDav protocol to sync calendar data. This app was subsequently open-sourced and available on {\href{https://github.com/dmfs/tasks}{Github}}.Was also part of the team which developed an Android based {\href{https://play.google.com/store/apps/details?id=com.schedjoules.calstore}{Calendar Store}}.
    
    But my primary role was as a Django developer. Developed a web service for the bulk-retail of mobile apps. The service used OpenSSL cryptography to sign and verify the source of the apps. Only apps with a valid key issued from the website could be activated. Also developed the Android SDK which integrates with third-party apps to provide this functionality.
}}
}

\headedsection
{\href{http://www.inquence.com/}{INQUENCE GmbH}}
{\textsc{Dresden, Germany}}
{
 \headedsubsection
 {\acr{Systems Developer}}
 {Mar 2014 - Jul 2014}
 {\bodytext{
INQUENCE develops a complete hardware and software solution for office for the the storage and search of the all types of physical and electronic documents. Wwas responsible for the maintenance of the web frontend for this solution, as well as the addition of new features. The software solution consisted of several services which were responsible for converting, indexing and storing the data. Developed a monitoring mechanism using Monit which kept track for all the individual hardware units and monitored for performance changes after software updates. 
}}
}

\headedsection
{\href{http://tu-dresden.de/die_tu_dresden/fakultaeten/fakultaet_informatik/sysa/se?set_language=en&cl=en}{TU Dresden}}
{\textsc{Dresden, Germany}}
{
 \headedsubsection
 {\acr{Wissenschaftliche Hilfskraft}}
 {Mar 2013 - Present}
 {\bodytext{
As a part of the {\href{http://tu-dresden.de/die_tu_dresden/fakultaeten/fakultaet_informatik/sysa/se/srex}{SREX(Secure Remote Execution)}} project developed a translator for C programs. The C programs were translated into a subset of C which can be encoded with redundant operations to ensure correct execution even in the event of bit-flips or other random transient errors.

More recently I have been working with distributed coordination services like ZooKeeper and Consul. I have been evaluating the efficacy and characteristics of the respective protocols(ZAB and Raft). Also included, is the implementation of new features in Kazoo, a Python client library for Zookeeper.
}}
}

\begin{center}
  \emph{Please consult my \href{https://de.linkedin.com/in/arnaik}{LinkedIn Profile} for references.}
\end{center}


\spacedhrule{-0.2em}{-0.4em}

\roottitle{Education}

\headedsection
  {BM Sreenivasaiah College of Engineering, Vivesvararya Technological University}
  {\textsc{Bangalore, India}} {%
  \headedsubsection
    {Bachelor of Computer Science and Engineering}
    {2006 -- 2010}
    {\bodytext{
        Major was in Computer Science with a focus on Computer Architecture. Was also one of editors of the departmental newsletter and also a regular contributor. For my final year project developed a Wi-Fi enable robot which could be controlled over the internet. It also had a plethora of sensors which guided in navigation and control.}
    }
}

\headedsection
  {Techniche Universit{\"a}t Dresden}
  {\textsc{Dresden, Germany}} {%
  \headedsubsection
    {Master of Science, Distributed Systems Engineering}
    {2012 -- 2014}
    {\bodytext{
        Completed courses in Security and Cryptography, Software Fault Tolerance and Dependable Systems other than the mandatory courses. Also worked in the Systems Engineering Lab and Software Fault Tolerance Lab. Under the guidance of Prof. Dr. Christof Fetzer Now in the process of completing the Master's Thesis. Implemented and evaluated a partitioned Zookeeper service to increase the throughput of Zookeeper without the negative effects of horizontal scaling.}
    }
}

\spacedhrule{1.6em}{-0.4em}

\roottitle{Projects}

\headedsection{
{\href{http://se.inf.tu-dresden.de/pubs/papers/Wamhoff2014a.pdf}{JoQuer: Turbo frequency scaling libraries}}
}
{\bodytext{Modern generation processors like the Intel Haswell series have the ability to temporarily boost the clock frequency by writing to certain registers. A C++ library was developed at the Systems Engineering chair at TU Dresden to provide this functionality to applications and manage them. I was given the responsibility of exposing this functionality through a Java library with a similar API through JNI(Java Native Interface). After the library was implemented, benchmarked and recorded the result to compare the effect of boosting frequency in Java applications.}
}

\headedsection{Studybloxx: Android list sync application
{\href{https://github.com/arjunrn/studybloxx-server}{(https://github.com/arjunrn/studybloxx-server)}}
}
{\bodytext{For a course in Mobile Application development developed an application for Android which syncs structured data across devices. The sync is performed with a Django web application through a \acr{REST API}. To further improve the application functionality LibGit2 has been cross-compiled to run on Android to sync the data. Git is used to handle the issues arising out of concurrent modification of data from multiple sources.}
}

\spacedhrule{0.5em}{-0.4em}

\roottitle{Skills}

\inlineheadsection  % special section that has an inline header with a 'hanging' paragraph
  {Technical Specialities:}
  {I consider myself language agnostic and I quickly learn and put to use any programming language which is required for the task at hand. However, Python is my favourite and go-to language for most problems. I have used many Python based libraries and frameworks over the course of career. I am very familiar with Django, Scrapy, Google AppEngine, Tornado, Fabric, pyOpenSSL. I have also a good understanding of Frontend web technologies: \acr{HTML+CSS}, JavaScript(jQuery, Angular.js). I have also some experience with developing web applications using PHP, Ruby and node.js. All my projects are managed through Git or Mercurial. I also have substantial experience developing beautiful native Android applications. Most of my knowledge of Java comes from developing Android applications. Recently I have dabbled in Go and Haskell to solve some concurrency based problems for my courses. I also have some experience using and tuning databases like MySQL, PostgreSQL, MongoDB and Redis. I have also been using Linux both professionally and for personal use for the past 6 year.
}

\inlineheadsection
  {Natural Languages:}
  {English \emph{(Bilingual Proficiency)}, Hindi \emph{(Native)}, Kannada \emph{(Native)},German \emph{(Limited proficiency)}.}


\spacedhrule{1.6em}{-0.4em}

\roottitle{Interests}

\inlineheadsection
  {Non-exhaustive and in no particular order:}
  {Science-Fiction, photography, hardware hacking, travelling, good coffee}


\end{document}
